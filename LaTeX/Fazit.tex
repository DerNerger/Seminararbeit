\chapter{Fazit und Ausblick}
Die durchgeführte Analyse lieferte Informationen über die Qualität des implementierten TrackCleaners. Wird als Gütekriterium die Differenz der fehlerhaft entfernten Hits und der korrekt entfernten Hits zu Grunde gelegt ergibt sich eine optimale Distanz von 0.6, da hier die Anzahl der korrekt aussortierten Hits die Anzahl der fehlerhaft aussortierten Hits um 21 übersteigt.
Eine Differenz von 21 stellt bei 1000 Simulierten Events jedoch keine signifikante Überzahl der korrekt entfernten Hits dar. Selbst bei optimaler Distanz liegt die Anzahl der korrekt entfernten Hits nur minimal über der Anzahl der fehlerhaft entfernten Hits. Dabei kommt noch erschwerend hinzu, dass im Zweifelsfall wünschenswerter ist einen Track mit falschen Hits zu rekonstruieren als korrekte Hits aus einem Track zu entfernen. Die hohe Anzahl fehlerhaft entfernter Hits ist darauf zurückzuführen, dass sich bei rekonstruierten Tracks mit einigen fehlerhaften Hits schlechte Approximation ergeben. Da die Approximation meist besonders schlecht ist, wenn die fehlerhaften Hits weit von den übrigen Hits entfernt liegen könnten in einer weiterführenden Arbeit diese Tracks erkannt und vom Bereinigen ausgeschlossen werden. Das Laufzeitverhalten des TrackCleaners ist hingegen positiver zu beurteilen. Das bereinigen der Tracks nach einer Rekonstruktion würde das Laufzeitverhalten nicht signifikant beeinflussen und ist somit für den Einsatz in einem Online-Tracker geeignet. Darüber hinaus wäre der TrackCleaner sehr einfach zu parallelisieren, da die Bereinigung eines Tracks völlig unabhängig von der Verarbeitung der übrigen Tracks funktioniert. Zusammenfassend lässt sich sagen, dass sich der TrackCleaner  ohne weitgehende Verbesserungen nur bedingt zum Bereinigen von Tracks einsetzen lässt, da die Anzahl der fehlerhaft entfernten Hits zu hoch ist.