\chapter{Einleitung}
Die Hadronenphysik beschäftigt sich mit der inneren Struktur von Atomkernen. Die Beobachtung komplexer Zerfallsketten liefern physikalisch interessante Erkenntnisse und stellen eine Möglichkeit dar, um Rückschlüsse auf die innere Teilchenstruktur ziehen zu können. Da die meisten solcher Ereignisse nicht stabil vorkommen werden Teilchenbeschleuniger eingesetzt, welche durch hohe Energiezuführung das Entstehen von physikalisch interessanten Teilchen herbeiführen. Bei der Konstruktion solcher Großgeräte handelt es sich um ein breites Aufgabenfeld. Zur physikalischen Analyse des Experiments sind die Kenngrößen Impuls, Energie, Art und Spur des Teilchens von Interesse. Zu deren Erfassung ist ein komplexes Detektorsystem nötig. Dieses besteht unter anderem aus Detektoren zur Rekonstruktion von Teilchenflugbahnen. Ein solcher spurgebender Detektor ist in der Lage einen sogenannten Hit zu melden, wenn ein Teilchen eine bestimmte Position passiert hat. Zur Auswertung dieser Detektoren werden Spurfindungs-Algorithmen eingesetzt. Diese versuchen aus den gemessenen Hits die Teilchenflugbahn zu rekonstruieren. Da es unmöglich ist einen perfekten Trackfinding-Algorithmus zu entwickeln, sind die rekonstruierten Tracks fehlerbehaftet. Somit ist es möglich, dass ein solcher Algorithmus Hits zu einem Track hinzufügt, welche in Wirklichkeit zu einem anderen physikalischen Track gehören. Um solche Tracks im Nachhinein von falschen Hits bestmöglich zu reinigen ist im Rahmen dieser Arbeit ein sogenannter Track-Cleaner entstanden. Zur Realisierung wurde für jeden Hit der Abstand zum approximierten Track bestimmt und ermittelt ob dieser Abstand einen parametrisierbaren Grenzwert $\mu$ überschreitet. Der Track-Cleaner ist im Rahmen des \pnd{}-Experiments entwickelt und getestet worden, kann jedoch prinzipiell für alle ähnlichen Experimente eingesetzt werden, da er unabhängig vom verwendeten Trackfinder und Detektorsystem funktioniert. Das \pnd{}-Experiment ist ein Teil der Teilchenbeschleunigeranlage FAIR, welche zur Zeit in Darmstadt entsteht. Im folgenden Kapitel wird zunächst auf FAIR eingegangen und das Detektorsystem \pnd{} erklärt. Anschließend wird das verwendete Framework Root erörtert und schließlich die Entwicklung des Trackfinders dokumentiert. Im letzten Kapitel wird erklärt, wie mittels einer Parameterstudie der optimale Grenzwert $\mu$ gefunden wurde.