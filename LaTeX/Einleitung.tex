\chapter{Einleitung}
Die Kernphysik beschäftigt sich unter anderem mit der inneren Struktur von Atomkernen. Um weitere Erkenntnisse auf diesem Gebiet zu erarbeiten wird meist auf Experimente zurückgegriffen. Die Beobachtungen komplexer Zerfallsketten liefern meist physikalisch interessante Erkenntnisse und stellen eine praktikable Möglichkeit dar, um Rückschlüsse auf die innere Teilchenstruktur ziehen zu können. Da die meisten solcher Ereignisse nicht natürlich vorkommen werden Teilchenbeschleuniger eingesetzt, um Teilchen durch extrem hohe Energiezuführung zum Zerfall zu bringen. Bei der Konstruktion solcher Großgeräte handelt es sich um ein breites Aufgabenfeld. Eine wichtige Aufgabe stellt dabei das Trackfinding dar, welches sich mit der Rekonstruktion von Teilchenflugbahnen beschäftigt. Diese werden als Tracks bezeichnet. Dazu werden komplexe Detektorsysteme konstruiert, welche in der Lage sind einen sogenannten Hit zu melden, wenn ein Teilchen eine bestimmte Position passiert. Zur Realisierung dieser Funktionalität sind verschiedene Ansätze denkbar. Entsprechende Algorithmen liefern dann aufgrund der gemessenen Hits einen rekonstruierten Track. Da es unmöglich ist einen perfekten Trackfinding-Algorithmus zu entwickeln, sind die rekonstruierten Tracks fehlerbehaftet.
Somit ist es möglich, dass ein solcher Algorithmus Hits zu einem Track hinzufügt, welche in Wirklichkeit zu einem anderen physikalischen Track gehören. Um solche Tracks im Nachhinein von falschen Hits bestmöglich zu reinigen ist im Rahmen dieser Arbeit ein sogenannter Track-Cleaner entstanden. Zur Realisierung wurde für jeden Hit der Abstand zum approximierten Track bestimmt und ermittelt ob dieser Abstand einen parametrisierbaren Grenzwert $\mu$ überschreitet. Der Track-Cleaner ist im Rahmen des \pnd{}-Experiments entwickelt und getestet worden, kann jedoch prinzipiell für alle ähnlichen Experimente eingesetzt werden, da er unabhängig vom verwendeten Trackfinder und Detektorsystem funktioniert. Das \pnd{}-Experiment ist ein Teil der Teilchenbeschleunigeranlage FAIR, welche zur Zeit in Darmstadt entsteht. Im folgenden Kapitel wird zunächst genauer auf FAIR eingegangen und das Detektorsystem \pnd{} erklärt. Anschließend wird das verwendete Framework Root erörtert und schließlich die Entwicklung des Trackfinders dokumentiert. Im letzten Kapitel wird erklärt, wie mittels einer Parameterstudie der optimale Grenzwert $\mu$ gefunden wurde.